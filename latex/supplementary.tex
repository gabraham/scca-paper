\documentclass[a4paper,12pt]{article}


%\usepackage{a4wide}
\usepackage{geometry}
\usepackage{amsmath}
\usepackage{amssymb}
\usepackage{hyperref}
\usepackage{graphicx}
\usepackage[numbers]{natbib}

\geometry{
   includeheadfoot,
   margin=2.54cm
}


\hypersetup{
   pdftitle={},
   pdfauthor={Gad Abraham},
   colorlinks=true,
   citecolor=black,
   filecolor=black,
   linkcolor=black,
   urlcolor=black
}

\newcommand{\nindiv}{601 }
\newcommand{\ngenes}{18,193 }
\newcommand{\nsnps}{1,125,747 }
\newcommand{\nsnpschr}{89,603 }


%\author{Gad Abraham and Michael Inouye}

\title{Fast sparse canonical correlation with flashpca --- Supplementary
Material}

\begin{document}

\maketitle

\section{Reproducibility}

Code to reproduce these experiments is at
\url{https://github.com/gabraham/scca-paper}.

\section{HapMap data preprocessing and quality control}

The HapMap3 phase III~\citep{hapmap2010} genotypes were obtained from
\url{ftp://ftp.ncbi.nlm.nih.gov/hapmap/genotypes/2009-01_phaseIII/plink_format/}.
Gene expression levels were obtained from
\url{http://www.ebi.ac.uk/arrayexpress/experiments/E-MTAB-264}.

We excluded individuals who were non-founders, had genotyping
missingness~${>}1\%$, or did not have matching gene expression data,
resulting in~602 individuals. We excluded non-autosomal SNPs, SNPs with
MAF~${<}5\%$, missingness~${>}1\%$, and deviation from Hardy-Weinberg
equilibrium~$P{<}5\times10^{-6}$ using PLINK~1.9~\citep{purcell2007,Chang2015},
leaving~\nsnps SNPs (see below). Missing genotypes were randomly imputed
according to the frequencies of the non-missing observations.

For the gene expression data, we used a subset consisting of the~21,800 probes
that were analysed by~\citep{Stranger2012}, utilising the original authors'
normalised data. Following~\citep{Stranger2012}, we performed PCA on the
genotypes within each population, and for the GIH, MEX, MKK, and LWK regressed
out~10 PCs of the genotypes (as well as intercept) from the corresponding gene
expression levels, in order to adjust for the higher levels of admixture within
these populations.  We further filtered probes with low variance
(std.~dev.~${<}0.1$), leaving~\ngenes probes. Both the gene expression levels and
the genotypes were standardised to zero-mean and unit-variance.

\section{Timing experiments}

For timing of \texttt{flashpcaR::scca} and \texttt{PMA::CCA}, we used contiguous
subsets of chromosome~1 (1000, 5000, 10,000, 20,000, and 50,000 SNPs, out
of~\ngenes SNPs in total) and contigous subsets of the~\ngenes gene expression
probes (100, 10,000, and all~\ngenes probes).

We used the \textsf{R} package \texttt{microbenchmark}~\citep{Mersmann2015} to
run~30 replications of each timing experiment. For \texttt{PMA::CCA}, we used
\texttt{flashpcaR::flashpca} to precompute the rank-1 singular value
decomposition of $\mathbf{X}^T \mathbf{Y}$ ($u_1 d_1 v_1^T$) in order to
initialise the CCA algorithm. For all experiments we estimated one pair of
canonical vectors $u_1, v_1$.

All experiments were run in \textsf{R}~3.2.2~\citep{R} on 64-bit Ubuntu
Linux~12.04 on an Intel Xeon CPU~E7-4830 v2 @~2.20GHz. Time for the commandline
\texttt{flashpca} include loading of data into RAM. We used flashpca~v1.2.5
(\url{https://github.com/gabraham/flashpca}) and PMA~v1.0.9~\citep{Witten2013}.

\section{Comparison of predictive power}

Utilising the chromosome~1 genotypes (\nsnpschr SNPs) and all~\ngenes gene
expression levels, we used 5-fold cross-validation to compare
\texttt{flashpcaR::scca} and \texttt{PMA::CCA}, over a~2D grid of $30\times25$
penalties, estimating one pair of canonical vectors. The final predictive power
was computed as the average Pearson correlation
$\bar{\rho}$ in the~$k=1,\hdots,5$ test folds:
$$
\bar{\rho} = \frac{1}{5} \sum_{k=1}^5
   \mbox{Cor}(\mathbf{X}_{test}^k u^k, \mathbf{Y}_{test}^k v^k).
$$

\bibliographystyle{unsrtnat}
\bibliography{paper}

\end{document}


\documentclass[a4paper,10pt]{article}


%\usepackage{a4wide}
\usepackage{geometry}
\usepackage{amsmath}
\usepackage{amssymb}
\usepackage{hyperref}
\usepackage{graphicx}
\usepackage[numbers]{natbib}

\geometry{
   includeheadfoot,
   margin=2.54cm
}


\hypersetup{
   pdftitle={},
   pdfauthor={Gad Abraham},
   colorlinks=true,
   citecolor=black,
   filecolor=black,
   linkcolor=black,
   urlcolor=black
}

\newcommand{\nindiv}{709 }
\newcommand{\ngenes}{18,379 }
\newcommand{\nsnps}{973,983 }
\newcommand{\nsnpschr}{79,011 }


\author{Gad Abraham and Michael Inouye}

\title{Fast sparse canonical correlation with flashpca --- Supplementary
Material}

\begin{document}

\maketitle

\section{Reproducibility}

Code to reproduce these experiments is at
\url{https://github.com/gabraham/scca-paper}.

\section{HapMap data preprocessing and quality control}

The HapMap3 phase III~\citep{hapmap2010} genotypes were obtained from
\url{ftp://ftp.ncbi.nlm.nih.gov/hapmap/genotypes/2009-01_phaseIII/plink_format/}.
Gene expression levels were obtained from
\url{http://www.ebi.ac.uk/arrayexpress/experiments/E-MTAB-264}.

We excluded individuals who were non-founders, had genotyping
missingness~${>}1\%$, or did not have matching gene expression data,
resulting in~\nindiv individuals. We excluded non-autosomal SNPs, SNPs
with MAF~${<}5\%$, missingness~${>}1\%$, and deviation from Hardy-Weinberg
equilibrium~$P{<}5\times10^{-6}$ using PLINK~1.9~\citep{purcell2007,Chang2015},
leaving~\nsnps autosomal SNPs. The remaining missing genotypes were randomly
imputed according to the frequencies of the non-missing observations.

For the gene expression data, we used a subset consisting of the~21,800 probes
that were analysed by~\citep{Stranger2012}, utilising the original authors'
normalised data. Following~\citep{Stranger2012}, we performed PCA on the
genotypes within each population, and for the GIH, MEX, MKK, and LWK regressed
out~10 PCs of the genotypes (as well as intercept) from the corresponding gene
expression levels, in order to adjust for the higher levels of admixture within
these populations.  We further filtered probes with low variance
(std.~dev.~${<}0.1$), leaving~\ngenes probes. Both the gene expression levels and
the genotypes were standardised to zero-mean and unit-variance.

\section{Comparison of predictive power with simulated gene expression data}
\label{section:sim}

The number of samples in the HapMap3 data ($n=601$) is not does provide
for adequate statistical power to detect weak correlations or difference
in correlations between two competing methods, particularly when 3-fold
cross-validation further reduces the sample size in the test data to
${\sim}200$. For example, there is only $30\%$ power to detect a correlation
$\rho=0.1$ at an $\alpha=0.05$ with 200 samples) at an $\alpha=0.05$ with~200
samples, and only $7\%$ power to detect a difference in correlations $\Delta
\rho=0.05$ at $\alpha=0.05$ (comparing correlations achieved by two methods).

Hence, simulate gene expression data with strong associations with the
genotypes, allowing for higher correlations to be observed and meaningfully
compared.  Utilising 10,000 SNPs from HapMap3 chromosome~1, we simulated 1,000 gene
expression levels as
$$
\mathbf{Y} = \mathbf{X} \mathbf{B} + \mathbf{E},
$$
where $\mathbf{X}$ are the genotypes ($n \times p$ matrix), $\mathbf{B}$
is a $p \times m$ matrix of weights, and $\mathbf{E}$ is an $n \times m$
matrix representing the error (noise). To match the sparsity assumptions
of SCCA, $\mathbf{B}$ was chosen to be a mixture of weights $\{0.001, 1\}$
with proportions 0.9999 and 0.0001 (across all $n \times m$ entries),
respectively.  Note that a value of 0.001 was used rather than zero, in
order prevent some probes from having zero genetic variance. Each column
$k=1,\hdots,m$ of $\mathbf{E}$ was $E_k \sim \mathcal{N}(0, \frac{1-h^2}{h^2}
\mbox{var}((\mathbf{X}\mathbf{B})_k))$, and $h^2=0.1$.

We used 3-fold cross-validation to compare \texttt{flashpcaR::scca} and
\texttt{PMA::CCA}, over a~2D grid of $30\times25$ penalties, estimating one
pair of canonical vectors. The final predictive power was computed as the
average Pearson correlation $\bar{\rho}$ in the~$k=1,\hdots,3$ test folds:
$$
\bar{\rho} = \frac{1}{3} \sum_{k=1}^3
   \mbox{Cor}(\mathbf{X}_{test}^k u^k, \mathbf{Y}_{test}^k v^k).
$$

The maximum of the average test Pearson correlation was identical for
\texttt{flashpcaR::scca} and \texttt{PMA::CCA} ($\rho{=}0.931$), completing
in~4m and~24m, respectively (parallelising over 3 cores).

\section{Timing experiments}

For timing of \texttt{flashpcaR::scca} and \texttt{PMA::CCA}, we used
contiguous subsets of HapMap3 chromosome~1 (1000, 5000, 10,000, 20,000,
and 50,000 SNPs, out of~\ngenes SNPs in total) and contiguous subsets of
the~\ngenes real gene expression probes (1000, 10,000, and all~\ngenes
probes from~\citep{Stranger2012}).

We used the \textsf{R} package \texttt{microbenchmark}~\citep{Mersmann2015}
to run~30 replications of each timing experiment.  For all
experiments we estimated one pair of canonical vectors $(u_1,
v_1)$.  For the results in the main text, we initialised (``warm
started'') $v_1$ to a standard normally-distributed vector of variates
$\sim\mathcal{N}(0,1)$. \texttt{PMA::CCA} and \texttt{flashpcaR::scca}
allow the user to provide their own initialisation\footnote{The commandline
version \texttt{flashpca} currently only supports random initialisation.},
and we experimented with other forms, including using the column means
of the gene expression data and the rank-1 singular value decomposition
(SVD) $\mathbf{X}^T \mathbf{Y} \approx u_1 d_1 v_1^T$). The overall trend of
\texttt{flashpcaR} being several-fold faster than \texttt{PMA} was consistent
across all three initialistion methods~(Figure~\ref{fig:s01}).

\begin{figure}[!tpb]
\centering
\includegraphics[width=\textwidth]{scca_timing_full-crop.pdf}
\caption{
Timing (median of 30 runs) of SCCA implemented in (i) the \texttt{flashpcaR}
(\textsf{R} package) and (ii) \texttt{flashpca} (stand-alone commandline tool),
compared with SCCA from \texttt{PMA}, using subsets of the HapMap3 dataset with real
gene expression levels as phenotypes. We compared three schemes for initialising
$v_1$: (i) ``mean'': column means of the gene expression data; (ii) ``rand'':
normally-distributed variates $\mathcal{N}(0, 1)$; and (iii) ``svd'': 1st
right singular value of $\mathbf{X}^T \mathbf{Y}$.
}
\label{fig:s01}
\end{figure}

All experiments were run in \textsf{R}~3.2.2~\citep{R} (with the original
LAPACK and BLAS libraries included in \textsf{R}) on 64-bit Ubuntu Linux~12.04
on an Intel Xeon CPU~E7-4830 v2 @~2.20GHz. Time for the commandline
\texttt{flashpca} include loading of data into RAM. We used flashpca~v1.2.6
(\url{https://github.com/gabraham/flashpca}) and PMA~v1.0.9~\citep{Witten2013}.
For \texttt{PMA::CCA}, we increased the maximum number of iterations to match
that used by \texttt{flashpcaR::scca} (default=1000), in order to prevent
early termination of the algorithm before adequate numerical convergence was
achieved. 

\section{Parallelising grid search for penalty optimisation}

As described in Section~\ref{section:sim}, using multiple
cores can speed up the penalty grid search for \texttt{PMA} and
\texttt{flashpcaR}. Within \textsf{R}, this can be achieved using the
\texttt{foreach}~\citep{foreach} and \texttt{doMC}~\citep{doMC} packages. We
recommend using coarse-grain parallelisation for cross-validation, e.g.,
5 cores for 5-fold cross-validation. Examples are given in the code at
\url{https://github.com/gabraham/scca-paper}.

The commandline tool \texttt{flashpca} currently does not support built-in
cross-validation (as of v1.2.6). We recommend splitting the data into
training/test folds using PLINK and running \texttt{flashpca} on these subsets,
possibly using GNU parallel~\citep{Tange2011a}.



\bibliographystyle{unsrtnat}
\bibliography{paper}

\end{document}

